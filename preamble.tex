\usepackage[a4paper,top=3cm,bottom=3cm,left=2.5cm,right=2.5cm]{geometry}
\usepackage{fancyhdr}
\fancyhf{} % limpia encabezados y pies
\fancyfoot[C]{\thepage} % número de página centrado en el pie
\pagestyle{fancy}
\usepackage{booktabs}
\usepackage{etoolbox}
\usepackage{hyperref}
\usepackage{longtable}
\usepackage{tabularx}
%%%%%%%%%%%%%%%%%%%%%%%%%%%%%%%%%%%%%%%%%%%%%%%%%%%%%%%%%%%%%%%%%%%%%%%
%%%%%%%%%%%%%%%%%%%%%%%%%%%%%%%%%%%%%%%%%%%%%%%%%%%%%%%%%%%%%%%%%%%%%%%
\usepackage{xcolor}
\definecolor{benettongreen}{RGB}{0, 173, 67}
\definecolor{mustardyellow}{RGB}{204, 143, 0}
\definecolor{darkgrey}{RGB}{60, 60, 60}
\definecolor{darkergrey}{rgb}{0.5, 0.5, 0.5}
\definecolor{lightgrey}{rgb}{0.8, 0.8, 0.8}
\definecolor{red}{RGB}{240, 0, 0}
\definecolor{blue}{RGB}{0, 0, 255}
\definecolor{acrcolor}{RGB}{150, 60, 0}
\definecolor{VeryDarkBlue}{rgb}{0.0, 0.0, 0.7}
\definecolor{DarkBloodRed}{RGB}{139, 0, 0}

\hypersetup{
    colorlinks=true,
    linkcolor=red,
    citecolor=VeryDarkBlue,
    filecolor=darkergrey,
    urlcolor=mustardyellow
}

%%%%%%%%%%%%%%%%%%%%%%%%%%%%%%%%%%%%%%%%%%%%%%%%%%%%%%%%%%%%%%%%%%%%%%%
%%%%%%%%%%%%%%%%%%%%%%%%%%%%%%%%%%%%%%%%%%%%%%%%%%%%%%%%%%%%%%%%%%%%%%%
\ifxetex
  \usepackage{polyglossia}
  \usepackage{fontspec}
  \setmainfont{Palatino}
  \setmainlanguage{spanish}
  % Tabla en lugar de cuadro
  \gappto\captionsspanish{\renewcommand{\tablename}{Tabla}
          \renewcommand{\listtablename}{Índice de tablas}}
\else
  \usepackage[spanish,es-tabla]{babel}
\fi

\AtBeginDocument{
    \AtBeginShipoutNext{
        \AtBeginShipoutUpperLeft{
            \put(\dimexpr\paperwidth/2-\textwidth/2\relax, -750){
                \makebox[\textwidth]{
                    \includegraphics[width=0.35\textwidth]{images/logomedia.png}  % Adjust width as needed
                    \hfill
                    \includegraphics[width=0.3\textwidth]{images/cure.png} % Make it 90% smaller
                }
            }
        }
    }
}

\usepackage{amsthm}
\makeatletter
\def\thm@space@setup{%
  \thm@preskip=8pt plus 2pt minus 4pt
  \thm@postskip=\thm@preskip
}

%%%%%%%%%%%%%%%%%%%%%%%%%%%%%%%%%%%%%%%%%%%%%%%%%%%%%%%%%%%%%%%%%%%%%%%
%%%%%%%%%%%%%%%%%%%%%%%%%%%%%%%%%%%%%%%%%%%%%%%%%%%%%%%%%%%%%%%%%%%%%%%
% --- Paquetes necesarios ---
\usepackage{amsthm}
\newtheoremstyle{miestilo}
  {10pt}    % espacio arriba
  {10pt}    % espacio abajo
  {\itshape}  % cuerpo del teorema
  {}        % sangría
  {\bfseries} % encabezado
  {.}       % puntuación
  { }       % espacio después del encabezado
  {}        % encabezado

\theoremstyle{miestilo}

\usepackage[most]{tcolorbox}
% --- Entornos tipo teorema en español ---
% --- Entornos matemáticos en español ---
\newtheorem{theorem}{Teorema}[chapter]
\newtheorem{definition}[theorem]{Definición}
\newtheorem{proposition}[theorem]{Proposición}
\newtheorem{lemma}[theorem]{Lema}
\newtheorem{corollary}[theorem]{Corolario}
\newtheorem{example}[theorem]{Ejemplo}
\newtheorem{exercise}[theorem]{Ejercicio}
\newtheorem{remark}[theorem]{Observación}
\newtheorem{solution}[theorem]{Solución}
% --- Estilos visuales para cada entorno ---
\tcolorboxenvironment{theorem}{
  colback=blue!5!white,
  colframe=blue!75!black,
  coltitle=black,
  fonttitle=\bfseries,
  breakable
}

\tcolorboxenvironment{definition}{
  colback=green!5!white,
  colframe=green!50!black,
  coltitle=black,
  fonttitle=\bfseries,
  breakable
}

\tcolorboxenvironment{proposition}{
  colback=cyan!5!white,
  colframe=cyan!60!black,
  coltitle=black,
  fonttitle=\bfseries,
  breakable
}







\tcolorboxenvironment{corollary}{
  colback=violet!10!white,
  colframe=violet!80!black,
  coltitle=black,
  fonttitle=\bfseries,
  breakable
}

%%%%%%%%%%%%%%%%%%%%%%%%%

